\documentclass[a4paper, 12pt]{article}

\usepackage[T1]{fontenc}
\usepackage[utf8]{inputenc}
\usepackage[french]{babel}
\usepackage{amsmath,amssymb,amsthm}
\usepackage{hyperref}
\usepackage{cleveref}
\usepackage{geometry}

\geometry{
    left=20mm,
    top=0mm,
}

\hypersetup{
    colorlinks=true,
    linkcolor=blue,
    filecolor=magenta,      
    urlcolor=cyan,
}

\title{Projet Analyse Numérique GM3 \\ {\Large Contribution à \emph{Control Toolbox} : Euler-Explicite}}
\author{T. Schmoderer\thanks{\href{mailto:timothee.schmoderer@insa-rouen.fr}{timothee.schmoderer@insa-rouen.fr}} \and A. Tonnoir\thanks{\href{mailto:antoine.tonnoir@insa-rouen.fr}{antoine.tonnoir@insa-rouen.fr}}}
\date{11 Janvier 2023}

\begin{document}
\maketitle

    La bibliothèque C\texttt{++} \emph{Control Toolbox} est développé dans le but de simuler de façon efficace les trajectoires des systèmes dynamique contrôlés. Dans ce cadre, L'\textbf{objectif} du projet est d'implémenter une méthode numérique de calcul des trajectoires d'un système dynamique classique 
    \begin{align}\label{eq-ode}
        \dot{x}(t) &= f(t,x), \quad x\in \mathbb{R}^n, \quad t\in[t_0,t_1],\quad x(t_0)=x_0.
    \end{align}
    \noindent
    Nous vous proposons d'implémenter la méthode d'\textbf{Euler-explicite} pour intégrer numériquement les ODEs du types \cref{eq-ode}.

\paragraph{Organisation du projet.}
    Le développement se fera sur le dépôt git de la bibliothèque : \href{https://github.com/tschmoderer/control-toolbox}{lien}. Pour mener le projet à son terme, nous organiserons une rencontre au début pour clarifier et définir les objectifs du projet. Si besoin, nous pourrons organiser une seconde rencontre pour débloquer d'éventuels problèmes. 

\paragraph{Travail attendu.}
    \begin{enumerate}
        \item Sur une nouvelle branche du dépôt, implémenter la méthode en C\texttt{++}, 
        \item Écrire la documentation de votre fonction, 
        \item Écrire un programme test,
        \item Écrire un programme d'exemple, 
        \item Écrire une pull-request vers le dépôt git pour intégrer votre travail. 
    \end{enumerate} 

\paragraph{Critères d'évaluation.}Le travail sera évalué sur la base des rencontres effectuées pendant la réalisation du projet, sur un court rapport contenant les éléments suivants: 

\begin{enumerate}
    \item Présentation générale: présentation claire et concise du problème, justification des méthodes employées.
    \item Programmation: Clarté et lisibilité du code, absence de calculs inutiles, rapidité d’exécution.
    \item Résultats numériques: Qualité des figures et leur analyse. 
    \item Conclusion: Mise en perspective du projet et votre formation en analyse numérique. 
\end{enumerate}

\end{document}